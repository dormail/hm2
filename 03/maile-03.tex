\documentclass[a4paper]{article}
\usepackage[utf8]{inputenc}
\usepackage[margin=2.5cm]{geometry}

% packages die oefters gebraucht werden
\usepackage{amsmath, amssymb}
\usepackage{graphicx}
\usepackage{fancyhdr}

\fancyhf{}
% vspaces in den headern fuer Distanzen notwendig
% linke Seite: Namen der Abgabegruppe
\lhead{\textbf{Matthias Maile}\vspace{0.5cm}}
% rechte Seite: Modul, Gruppe, Semester
\rhead{\textbf{Höhere Mathematik II\\Sommersemester 2020}\vspace{0.5cm}}
% Center: nr. des blattes
\chead{\vspace{1.5cm}\huge{\textbf{2. Übungsblatt}}}
% benoetigt damit der eigentliche Text nicht in der Überschrift steckt
\setlength{\headheight}{3cm} 

\begin{document}
\thispagestyle{fancy}
\section*{Aufgabe 9}
\par{a) i)}
\begin{align*}
	\int_0^3 \left(e^{3x} - \sqrt[3]{e^x} \right) dx
	&= \int_0^3 \left(e^{3x} - e^{\frac x3} \right) dx \\
	% integral trennen
	&= \int_0^3 e^{3x} dx - \int_0^3 e^{\frac x3} dx \\
	% aus produktregel folgt
	&= \left[ \frac13 e^{3x} + C \right]_0^3 -
	\left[3 e^{\frac x 3} + D \right]_0^3 \\
	% einsetzen
	&= \left( \frac{e^9}{3} + C - \frac{e^0}{3} - C \right) -
	\left(3e + D - 3e^0 - D \right) \\
	% kuerzen
	&= \frac{e^9}{3} - \frac13 - 3e + 3 \\
	% alles in einen bruch
	&= \frac{e^9 - 9e +8}{3}
\end{align*}
\par{ii)}
\begin{align*}
	\int_{\frac\pi6}^{\frac\pi3} \cot(x) \ln(\sin x) dx
	&= \int_{\frac\pi6}^{\frac\pi3} 
	\frac{\cos x}{\sin x} \ln(\sin x) dx \\
	% substitution
	u := \sin(x) \Rightarrow \frac{du}{dx} = \cos(x) 
	\Rightarrow dx = \frac{du}{\cos(x)}
	\Rightarrow \
	&= \int_{\sin\frac\pi6}^{\sin\frac\pi3} 
	\frac{\cos x}{\sin x} \ln(\sin x) \frac{du}{\cos x} \\
	% kuerzen
	&= \int_{\sin\frac\pi6}^{\sin\frac\pi3}
	\frac1{u} \ln(u) du \\
	% kettenregel
	\int g(x) g^\prime(x) dx \overset{KR}{=} 
	\frac12\left(g(x)\right)^2 + C
	\Rightarrow \
	&= \frac12 \left(\ln(u)\right)^2 + C 
	\vert_{\frac12}^{\frac{\sqrt{3}}{2}}
\end{align*}



\section*{Aufgabe 10}
\end{document}
