\documentclass[a4paper]{article}
\usepackage[utf8]{inputenc}
\usepackage[margin=2.5cm]{geometry}

% packages die oefters gebraucht werden
\usepackage{amsmath, amssymb}
\usepackage{graphicx}
\usepackage{fancyhdr}

\fancyhf{}
% vspaces in den headern fuer Distanzen notwendig
% linke Seite: Namen der Abgabegruppe
\lhead{\textbf{Matthias Maile}\vspace{0.5cm}}
% rechte Seite: Modul, Gruppe, Semester
\rhead{\textbf{Höhere Mathematik II\\Sommersemester 2020}\vspace{0.5cm}}
% Center: nr. des blattes
\chead{\vspace{1.5cm}\huge{\textbf{2. Übungsblatt}}}
% benoetigt damit der eigentliche Text nicht in der Überschrift steckt
\setlength{\headheight}{3cm} 

\begin{document}
\thispagestyle{fancy}
\section*{Aufgabe 5}
\par{a)}
\[
	\sum_{n=0}^\infty \frac{n^2}{1+n^3} (x-1)^n 
\]
Der 0. Koeffizient $a_0$ ist 0, daher kann der Index verschoben werden:
\begin{align*}
	\sum_{n=0}^\infty \frac{n^2}{1+n^3} (x-1)^n =
	\sum_{n=1}^\infty \frac{n^2}{1+n^3} (x-1)^n
\end{align*}
Da $a_n = \frac{n^2}{1+n^3}>0$ für $n\in\mathbb{N}$ gilt, kann man die
Quotientenformel für den Konvergenzradius benutzen:
\begin{align*}
	r = \frac 1 c \text{ für }
	c &= \lim_{n\rightarrow\infty}
	\frac{\vert a_{n+1} \vert}{\vert a_n \vert} \\
	% einsetzen
	&= \lim_{n\rightarrow\infty} 
	\frac{(n+1)^2}{1 + (n+1)^3}
	\frac{1 + n^3}{n^2} \\
	% binomische formeln
	&= \lim_{n\rightarrow\infty} 
	\frac{n^2 + 2n + 1}{1 + n^3 + 3n^2 + 3n + 1}
	\frac{1+n^3}{n^2}\\
	% aus multiplizieren
	&= \lim_{n\rightarrow\infty} 
	\frac{n^5 + 2n^4 + n^3 + n^2 + 2n + 1}
	{n^5 + 3n^4 + 3n^3 + 2n^2} \\
	% kuerzen
	&= \lim_{n\rightarrow\infty} 
	\frac{1 + 2n^{-1} + n^{-2} + n^{-3} + 2n^{-4}+ n^{-5}}
	{1 + 3n^{-1} + 3n^{-2} + 2n^{-3}} = 1 \\
	\Rightarrow
	r &= \frac1c = 1
\end{align*}
% Randpunkte
Die Betrachtung der Randpunkte zeigte dazu:
\begin{itemize}
	\item Für $x=0$:
		$ \sum_{n=1}^\infty \frac{(-1)^n}{n} 
		\Rightarrow$
		alternierende Nullfolge, konv. nach Leibnizkriterium
	\item Für $x=2$:
		$ \sum_{n=1}^\infty \frac{1}{n}
		\Rightarrow $
		divergiert
\end{itemize}
Damit konvergiert die Potenzreihe für $ x \in [0, 2)$

\newpage
\setlength{\headheight}{0cm}
\par{b)}
\[
	\sum_{n=0}^\infty \frac{n! 2^{2n}}{(2n)!} (x - 1)^n
\]
da alle Koeffizienten $\neq 0$ sind, kann die Formel 
$r = \frac 1 {\lim_{n\rightarrow\infty} \frac{\vert a_{n+1} \vert}{\vert a_n \vert}} $ benutzt werden:
\begin{align*}
	r = \frac 1 c 
	\text{ mit }c &= \lim_{n\rightarrow\infty} 
	\frac{\vert a_{n+1} \vert}{\vert a_n\vert} \\
	% einsetzen
	&= \lim_{n\rightarrow\infty}
	\frac{(n+1)! \ 2^{2n+2}}{(2n+2)!}
	\frac{(2n)!}{n! \ 2^{2n}} \\
	% kuerzen
	&= \lim_{n\rightarrow\infty}
	\frac{4 (n+1)}{(2n+2)(2n+1)} \\
	% weiter kuerzen
	&= \lim_{n\rightarrow\infty}
	\frac{2}{2n+1}
	=
	2 \\
	% in r einsetzen
	\Rightarrow r &= \frac{1}{2}
\end{align*}
Die Konvergenz an den Randpunkten folgt aus dem Quotientenkriterium 
(aufgrund der Betragsstriche ist dabei die Betrachtung von 
$x = x_0 - 0.5$ mit der von $x = x_0 + 0.5$ identisch.
\begin{align*}
	\left|
	\frac{a_{n+1}}{a_n}\right| 
	&= \left|
	\frac{(n+1)! \ 2^{2n+2} \ (\pm0.5)^{n+1}}{(2n+2)!} 
	\frac{2n!}{n! * 2^{2n} * (\pm0.5)^{n+1}}
	\right| 
	\overset ! < 1
	\\ 
	&= \left|
	\frac{(n+1) * 4 * (\pm0.5)}{(2n+2)(2n+1)}
	\right| \\
	&= \left|
	\frac{\pm2}{2*(2n+1)} 
	\right| = \frac 1 {2n + 1} < 1 \quad
	\text{für } n \in \mathbb{N}
\end{align*}
Durch die Konvergenz and den Rändern des Intervalls besitzt die Reihe das Konvergenzintervall 
\[ x \in \left[\frac12, \frac32 \right]. \]

\par{c)}
\[
	\sum_{n=0}^\infty n^22^n \left(x + \frac 1 4\right)^n
	\]
Die Berechnungsformel für den Konvergenzradius:
\begin{align*}
	r = \frac 1 c \text{ mit }
	c &= \limsup_{n\rightarrow\infty} \sqrt[n]{\vert a_n \vert} \\
	% einsetzen
	&= \limsup_{n\rightarrow\infty} \sqrt[n]{n^2 2^n} \\
	% 2^n rausziehen
	&= \limsup_{n\rightarrow\infty} 2 \sqrt[n]{n^2} \\
	% quadrat rausziehen
	&= \limsup_{n\rightarrow\infty} 2
	\left( \sqrt[n]{n} \right)^2 = 2 \\
	\Rightarrow
	r &= \frac 1 2, \quad x_0 = -\frac14
\end{align*}
Für die Randpunkte:
\begin{itemize}
	\item $x = -\frac12 + x_0 = -\frac34$: $\sum_{n=0}^\infty n^22^n 
		\left( -\frac12 \right)^n = 
		\sum_{n=0}^\infty n^2 (-1)^n
		\Rightarrow
		$
		keine Nullfolge, somit divergiert die Reihe bei 
		$x = -\frac34$
	% obere schranke
	\item $x = \frac12 + x_0 = \frac14$: $\sum_{n=0}^\infty n^22^n
		\left( \frac14 + \frac14 \right)^n
		=
		\sum_{n=0}^\infty n^22^n \left( \frac12 \right)^n
		=
		\sum_{n=0}^\infty n^2
		\Rightarrow
		$
		keine Nullfolge, die Reihe divergiert bei 
		$x = \frac14$
\end{itemize}
Die Reihe Konvergiert also für $x \in \left(-\frac34, \frac14\right)$

\vspace{0.5cm}
\setlength{\headheight}{0cm} 
\section*{Aufgabe 6}
\[
	f: \mathbb{R} \rightarrow \mathbb{R}, \quad
	f(x) = (1 + x) e^x
\]
Für eine Funktion der Form $(a+x) e^x$ folgt die Ableitung aus der Produktregel:
\[ 
	\frac{d}{dx} \left( (a+x)e^x \right) 
	= (a+x) * \frac{d}{dx} e^x + \left( \frac{d}{dx} (a+x) \right) e^x
	= (a+x) e^x + 1 * e^x 
	= (a+1+x) e^x
\]
Die $n$-te Ableitung von $f$ lautet demnach:
\[
	f^{(n)} (x) = (1+n+x) e^x
\]
Am Entwicklungspunkt $x_0 = -1$ lautet diese:
\[
	f^{(n)} (-1) = n * e^{-1} = \frac n e
\]
Aus den Ableitungen folgt die Taylorreihe:
\[
	T_\infty(x; -1) 
	= \sum_{i=0}^\infty \frac{f^{(i)}(-1)}{i!} * (x+1)^i
	= \sum_{i=0}^\infty \frac{i}{e * i!} * (x+1)^i
\]
Die Koeffizienten $a_n$ des Taylorpolynoms sind für $n > 0$ positiv, 
da der Funktionswert am Entwicklungspunkt 0 ist, kann das 0. Taylorpolynom
vernachlässigt werden.\\
Der Konvergenzradius folgt dann:
\begin{align*}
	r = \frac 1 c \text{ mit } 
	c &= \lim_{n\rightarrow\infty} \frac{\vert a_{n+1} \vert}{\vert a_n \vert} \\
	&= \lim_{n\rightarrow\infty} 
	\frac{n+1}{e * (n+1)!} * \frac{e * n!}{n} \\
	% kuerzen
	&= \lim_{n\rightarrow\infty}
	\frac{(n+1) * n!}{(n+1)! \ n} \\
	% weiter kuerzen
	&= \lim_{n\rightarrow\infty}
	\frac1 n = 0 \\
	\Rightarrow
	r &= \frac{1}{c} = \infty
\end{align*}
Da der Konvergenzradius vom Taylorpolynom gegen $\infty$ geht, stellt es die Funktion $f$ 
auf ganz $\mathbb{R}$ dar.
\end{document}
