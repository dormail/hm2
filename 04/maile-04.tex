	\documentclass[10pt]{article}
\usepackage[utf8]{inputenc}
\usepackage[margin=2.5cm]{geometry}

% packages die oefters gebraucht werden
\usepackage{amsmath, amssymb}
\usepackage{graphicx}
\usepackage{fancyhdr}
\usepackage{aligned-overset}

\fancyhf{}
% vspaces in den headern fuer Distanzen notwendig
% linke Seite: Namen der Abgabegruppe
\lhead{\textbf{Matthias Maile}\vspace{0.5cm}}
% rechte Seite: Modul, Gruppe, Semester
\rhead{\textbf{Höhere Mathematik II\\Sommersemester 2020}\vspace{0.5cm}}
% Center: nr. des blattes
\chead{\vspace{1.5cm}\huge{\textbf{4. Übungsblatt}}}
% benoetigt damit der eigentliche Text nicht in der Überschrift steckt
\setlength{\headheight}{3cm} 

% custom commands
\newcommand{\arsinh}{\text{arsinh}}

\begin{document}
\thispagestyle{fancy}
\section*{Aufgabe 13}
a)
Faktorisierung der Funktion:
\begin{align*}
	\frac{x}{(x+1)^3} &=
	\frac{A}{x+1} + \frac{B}{(x+1)^2} + \frac{C}{(x+1)^3} \\
	% umstellen
	\Leftrightarrow
	x &= 
	A (x+1)^2 + B (x+1) + C 
	\quad \Rightarrow
	A=0 \ 
	^\text{Für gleichheit beider Seiten}
	_\text{muss Grad gleich sein} \\
	% weiter umstellen
	\Rightarrow x &=
	B \cdot x + B + C
	\quad \Rightarrow B = -C, \ B = 1 \Rightarrow C = -1 \\
	\Rightarrow
	\frac{x}{(x+1)^3} &=
	\frac{1}{(x+1)^2} - \frac{1}{(x+1)^3}
\end{align*}

Dann lässt sich das Integral evaluieren:
\begin{align*}
	\int_0^\infty \frac{x}{(x+1)^3} \ dx
	&= 
	\int_0^\infty \left( 
	\frac{1}{(x+1)^2} - \frac{1}{(x+1)^3}
	\right)
	\ dx \\
	% integral trennen
	&= 
	\int_0^\infty \frac{1}{(x+1)^2} \ dx -
	\int_0^\infty \frac{1}{(x+1)^3} \ dx \\
	% stammfkt einsetzen
	\text{Skript S.346} \Rightarrow
	&=
	\left[ -\frac{1}{x+1} + c\right]_0^\infty
	-
	\left[ -\frac1{2 (x+1)^2} + d \right]_0^\infty \\
	% werte einsetzen
	&= -0 + 1 + 0 - \frac12 \\
	&= \frac12
\end{align*}
b)
\begin{align*}
	&\int_1^\infty \frac{x}{\sqrt{x^4 + 1}} \ dx 
	\hspace{1cm} u:= x^2 \Rightarrow dx = \frac{du}{2x} \\
	% erste substitution
	= &\int_1^\infty \frac{1}{\sqrt{u^2 + 1}} \ du \\
	% 0.5 rausziehen
	= &\frac12 \int_1^\infty \frac{1}{\sqrt{u^2 + 1}} \ du \\
	= &\frac12 \left[\arsinh(u) + c \right]_0^\infty \\
	\Rightarrow & \arsinh(u) \text{ divergiert für } u \rightarrow
	\infty
\end{align*}

Somit existiert das Integral nicht.
\newpage
\setlength{\headheight}{0cm}
c)
\begin{align*}
	&\int_0^1 \frac{1}{e^x - 1} \ dx \quad
	u := e^x \Rightarrow dx = \frac{du}{e^x} = \frac{du}{u} \\
	% substitution einsetzen
	=& \int_1^e \frac{du}{u \cdot (u-1)} 
\end{align*}

Dieser Bruch kann mit einer Partialbruchzerlegung ausgewertet werden:
\begin{align*}
	\frac{1}{u \cdot (u-1)} &= \frac{A}{u-1} + \frac{B}{u} \\
	\Leftrightarrow
	1 &= A \cdot u + B \cdot (u-1) \ \Rightarrow
	A = 1, \ B = -1
\end{align*}

Einsetzen ins Integral:
\begin{align*}
	\int_1^e \frac{du}{u \cdot (u-1)} 
	&= 
	\int_1^e \frac{du}{u-1} - \int_1^e \frac{du}{u}  \\
	% stammfkt bilden
	&= \left[ \ln\vert u-1 \vert - \ln(u) \right]_1^e \\
	% ln zusammen ziehen
	&= \ln\left(\frac{u-1}{u}\right)_1^e \\
	% einsetzen
	&= \ln\left(\frac{e-1}{e}\right) -
	\underbrace{\ln\left(\frac{1-1}{1}\right)}_{\ln0 \rightarrow -\infty}
\end{align*}
Da im letzten Schritt der $\ln 0$ bestimmt werden soll (oder durch eine 
andere Darstellung eine 0-Division auftritt, ist das Integral nicht 
definiert.



\section*{Aufgabe 14}

a) Gegeben war die Reihe
\[
	\sum_2^\infty \frac{1}{n \ln(n) \ln(\ln(n))}
\]
Mit dem Integralkriterium lässt sich die Divergenz feststellen:
\begin{align*}
	&\int_{n_0}^\infty \frac{1}{n \ln(n) \ln(\ln(n))} \ dn \hspace{1cm}
	u := \ln(n) \Rightarrow dn = n \cdot du \\
	% 1. subtistution einsetzen
	\Rightarrow 
	= &\int_{\ln n_0}^\infty \frac{n \cdot du}{n \cdot u \cdot \ln(u)} 
	\hspace{1.5cm} z := \ln(u) \Rightarrow du = u \cdot dz \\
	% zweite substitution einsetzen
	\Rightarrow
	= &\int_{\ln(\ln n_0)}^\infty \frac{u \ dz}{u \ln(z)} \\
	% kuerzen
	= &\int_{\ln(\ln n_0)}^\infty \frac{dz}{z} \\
	% hauptsatz
	\text{Hauptsatz } \Rightarrow =
	& \left[\ln z \right]_{\ln(\ln n_0)}^\infty = \infty - \ln(\ln n_0)
	= \infty
\end{align*}

Da das unbestimmte Integral für alle $n_0$ divergiert muss auch die Reihe 
divergieren. \newline
\vspace{0.5cm}
\newline
\newpage
b) Gegeben war die Reihe 
\[
	\sum_2^\infty \frac1{n \ln(n) \ln^2(\ln n)}
\]

Mit dem Integralkriterium lässt sich zeigen:
\begin{align*}
	&\int_{n_0}^\infty \frac{1}{n \ln(n) \ln^2(\ln(n))} \ dn 
	\hspace{1cm}
	u := \ln(n) \Rightarrow dn = n \cdot du \\
	% 1. subtistution einsetzen
	\Rightarrow 
	= &\int_{\ln n_0}^\infty \frac{n \cdot du}{n \cdot u \cdot 
	\ln^2(u)} 
	\hspace{1.5cm} z := \ln(u) \Rightarrow du = u \cdot dz \\
	% zweite substitution einsetzen
	\Rightarrow
	= &\int_{\ln(\ln n_0)}^\infty \frac{u \ dz}{u \ln^2(z)} \\
	% kuerzen
	= &\int_{\ln(\ln n_0)}^\infty \frac{dz}{z^2} \\
	% hauptsatz
	\text{Hauptsatz } \Rightarrow =
	& \left[\frac{-1}{z} \right]_{\ln(\ln n_0)}^\infty 
	=
	0 - \frac{-1}{\ln(\ln(n_0))} = \frac{1}{\ln(\ln(n_0))}
\end{align*}
Da das Integral für $n_0 > 1$  bestimmt ist, ist die Reihe konvergent.
\end{document}
