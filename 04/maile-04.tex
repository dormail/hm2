	\documentclass[10pt]{article}
\usepackage[utf8]{inputenc}
\usepackage[margin=2.5cm]{geometry}

% packages die oefters gebraucht werden
\usepackage{amsmath, amssymb}
\usepackage{graphicx}
\usepackage{fancyhdr}
\usepackage{aligned-overset}

\fancyhf{}
% vspaces in den headern fuer Distanzen notwendig
% linke Seite: Namen der Abgabegruppe
\lhead{\textbf{Matthias Maile}\vspace{0.5cm}}
% rechte Seite: Modul, Gruppe, Semester
\rhead{\textbf{Höhere Mathematik II\\Sommersemester 2020}\vspace{0.5cm}}
% Center: nr. des blattes
\chead{\vspace{1.5cm}\huge{\textbf{3. Übungsblatt}}}
% benoetigt damit der eigentliche Text nicht in der Überschrift steckt
\setlength{\headheight}{3cm} 

\begin{document}
\thispagestyle{fancy}
\section*{Aufgabe 13}
a)
Faktorisierung der Funktion:
\begin{align*}
	\frac{x}{(x+1)^3} &=
	\frac{A}{x+1} + \frac{B}{(x+1)^2} + \frac{C}{(x+1)^3} \\
	% umstellen
	\Leftrightarrow
	x &= 
	A (x+1)^2 + B (x+1) + C 
	\quad \Rightarrow
	A=0 \ 
	^\text{Für gleichheit beider Seiten}
	_\text{muss Grad gleich sein} \\
	% weiter umstellen
	\Rightarrow x &=
	B \cdot x + B + C
	\quad \Rightarrow B = -C, \ B = 1 \Rightarrow C = -1 \\
	\Rightarrow
	\frac{x}{(x+1)^3} &=
	\frac{1}{(x+1)^2} - \frac{1}{(x+1)^3}
\end{align*}
Dann lässt sich das Integral evaluieren:
\begin{align*}
	\int_0^\infty \frac{x}{(x+1)^3} \ dx
	&= 
	\int_0^\infty \left( 
	\frac{1}{(x+1)^2} - \frac{1}{(x+1)^3}
	\right)
	\ dx \\
	% integral trennen
	&= 
	\int_0^\infty \frac{1}{(x+1)^2} \ dx -
	\int_0^\infty \frac{1}{(x+1)^3} \ dx \\
	% stammfkt einsetzen
	\text{Skript S.346} \Rightarrow
	&=
	\left[ -\frac{1}{x+1} + c\right]_0^\infty
	-
	\left[ -\frac1{2 (x+1)^2} + d \right]_0^\infty \\
	% werte einsetzen
	&= -0 + 1 + 0 - \frac12 \\
	&= \frac12
\end{align*}



\newpage
\setlength{\headheight}{0cm}

\section*{Aufgabe 14}

\end{document}
