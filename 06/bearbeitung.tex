\documentclass[10pt]{article}
\usepackage[utf8]{inputenc}
\usepackage[margin=2.5cm]{geometry}

% packages die oefters gebraucht werden
\usepackage{amsmath, amssymb}
\usepackage{graphicx}
\usepackage{fancyhdr}
\usepackage{aligned-overset}
\usepackage{float}

\fancyhf{}
% vspaces in den headern fuer Distanzen notwendig
% linke Seite: Namen der Abgabegruppe
\lhead{\textbf{Matthias Maile}\vspace{0.5cm}}
% rechte Seite: Modul, Gruppe, Semester
\rhead{\textbf{Höhere Mathematik II\\Sommersemester 2020}\vspace{0.5cm}}
% Center: nr. des blattes
\chead{\vspace{1.5cm}\huge{\textbf{6. Übungsblatt}}}
% benoetigt damit der eigentliche Text nicht in der Überschrift steckt
\setlength{\headheight}{3cm} 

% custom commands
\newcommand{\arsinh}{\text{arsinh}}

% for newpage to reset headheight
\newcommand{\secondpage}{
	\newpage 
	\setlength{\headheight}{0cm}
}

\begin{document}
\thispagestyle{fancy}

\section*{Aufgabe 21}
Gegeben sei die Funktion 
\[
	f(x,y) = e^{x-y} \left(\frac{x^3}{3} - y\right)
\]
dann folgen die partiellen Ableitungen aus der Produkt- und Kettenregel und
dem Satz von Schwartz:
\begin{align*}
	% erste ableitung nach x
	\partial_x f(x,y) 
	&= e^{x-y} \left(\frac{x^3}{3} - y + x^2 \right) \\
	% erste ableitung nach y
	\partial_y f(x,y) 
	&= e^{x-y} \left( y - \frac{x^3}3 - 1\right) \\
	% zweite nach x
	\partial^2_x f(x,y) 
	&= e^{x-y} \left(\frac{x^3}{3} - y + 2x^2 + 2x\right) \\
	% zweite nach y
	\partial^2_y f(x,y) 
	&= e^{x-y} \left(2 -y + \frac{x^3}3 \right) \\
	% ableitung nach x und y
	\partial_x\partial_y f(x,y) &= 
	\partial_y\partial_x f(x,y)
	= e^{x-y} \left( y - \frac{x^3}3 - 1 - x^2\right) \\
\end{align*}
Das notwendige Kriterium für Extrema:
\[ \text{grad} f(\vec a) = \vec 0 \]
Der Gradient folgt aus den zuvor berechneten Ableitungen:
\begin{align*}
	\text{grad} f(x,y) 
	&= \left( \partial_x f(x,y), \ \partial_y f(x,y) \right) \\
	% ableitung einsetzen
	&= \left( e^{x-y} \left(\frac{x^3}{3} - y + x^2 \right), \
		e^{x-y} \left( y - \frac{x^3}3 - 1\right) \right) 
	\overset != \vec 0 = (0,0)
\end{align*}
% unterbrechung um alignment neu zu setzen
\begin{align*}
	\Leftrightarrow
	e^{x-y} \left(\frac{x^3}{3} - y + x^2 \right) &= 0 
	&
	& \land
	&e^{x-y} \left( y - \frac{x^3}3 - 1\right) &= 0 \\
	% e funktion rauskuerzen
	\Leftrightarrow
	\frac{x^3}{3} - y + x^2 &= 0 
	& &&
	y - \frac{x^3}3 - 1 &= 0 \\
	% e funktion rauskuerzen
	\Rightarrow
	\frac{x^3}{3} + x^2 &= y \quad (21.1)
	& &&
	^\text{(21.1)}_\text{einsetzen} \Rightarrow
	\frac{x^3}{3} + x^2 - \frac{x^3}3 - 1 &= 0 \\
	% nach x umstellen
	^\text{21.2}_\text{einsetzen} \Rightarrow
	\frac{(\pm 1)^3}{3} + (\pm 1)^2 &= y
	&&&
	\Rightarrow x &= \pm 1 \quad (21.2) \\
	% y ausrechnen
	\Leftrightarrow 
	1 \pm \frac13 = y
\end{align*}
Damit lauten die zwei möglichen Extremstellen
\[
	\vec a_1 = \left( 1, \ \frac 43 \right) \quad
	\vec a_2 = \left( -1, \ \frac 23 \right)
\]

\secondpage

Um zu bestimmen ob es sich um Maxima oder Minima (oder Sattelstellen)
handelt betrachten wir die Definitheit der Hessematrix an den mögl. 
Extremstellen.\\
Die Hessematrix folgt aus den Ableitungen:
\[
	H_f(\vec a)
	= \begin{pmatrix}
		\partial_x\partial_x f(x,y) &
		\partial_x\partial_y f(x,y) \\
		\partial_y\partial_x f(x,y) &
		\partial_y\partial_y f(x,y)
	\end{pmatrix}
	= \begin{pmatrix}
	e^{x-y} \left(\frac{x^3}{3} - y + 2x^2 + 2x\right) &
	e^{x-y} \left( y - \frac{x^3}3 - 1 - x^2\right) \\
	e^{x-y} \left( y - \frac{x^3}3 - 1 - x^2\right) &
	e^{x-y} \left(2 -y + \frac{x^3}3 \right)
	\end{pmatrix}
\]
Für $\vec a_1$ finden wir dann ein relatives Minimum (mit dem hinreichenden Kriterium):
\begin{align*}
	\det \left(H_f\left(1, \ \frac 43 \right) \right)
	&= \det
	\begin{pmatrix}
	e^{-\frac13} \left(\frac{1}{3} - \frac43 + 2 + 2\right) &
	e^{-\frac13} \left( \frac43 - \frac{1}3 - 2\right) \\
	e^{-\frac13} \left( \frac43 - \frac{1}3 - 2 \right) &
	e^{-\frac13} \left(2 - \frac43 + \frac{1}3 \right)
	\end{pmatrix} \\
	% e raus ziehen
	&= \det
	\begin{pmatrix}
	3 \cdot e^{-\frac13}&
	-e^{-\frac13} \\
	-e^{-\frac13} &
	e^{-\frac13}
	\end{pmatrix} \\
	&= 3 \cdot e^{-\frac{2}{3}} - e^{-\frac{2}{3}} \\
	&= 2 \cdot e^{-\frac23} > 0 \\
	% 2. abletung einer richtung
	\partial_x\partial_x f\left(1, \ \frac 43 \right) 
	&= e^{-\frac13} \left(\frac{1}{3} - \frac43 + 2 + 2\right) \\
	&= 3 \cdot e^{-\frac13} > 0
\end{align*}
Mit dem hinreichenden Kriterium finden wir dann einen Sattelpunkt bei
$\vec a_2$:
\begin{align*}
	\det \left(H_f\left(-1, \ \frac 23 \right) \right)
	&= \det
	\begin{pmatrix}
	e^{-\frac53} \left(\frac{(-1)^3}{3} - \frac23 + 
			2 \cdot (-1)^2 + 2 \cdot (-1) \right) &
	e^{-\frac53} \left( \frac23 - \frac{(-1)^3}3 - 1 - (-1)^2\right) \\
	e^{-\frac53} \left( \frac23 - \frac{(-1)^3}3 - 1 - (-1)^2\right) &
	e^{-\frac53} \left(2 - \frac23 + \frac{(-1)^3}3 \right)
	\end{pmatrix} \\
	&= \det
	\begin{pmatrix}
	-e^{-\frac53} &
	-e^{-\frac53} \\
	-e^{-\frac53} &
	e^{-\frac53} 
	\end{pmatrix} \\
	&= -e^{-\frac{10}{3}} - e^{-\frac{10}{3}} \\
	&= -e^{-\frac{10}{3}} < 0
\end{align*}

\newpage
	
\section*{Aufgabe 22}
b) Eine Einschränkung der $\varphi$-Komponente liefert uns eine 
globale bijektive Abbildung, dazu wird nicht auf die $z$-Achse 
abgebildet:
\[
	\vec \phi: (0,\infty) \times [0, 2\pi) \times \mathbb R 
	\rightarrow \mathbb R^3 \setminus(z-\text{Achse})
	, 
	\hspace{1cm}
	\vec \phi (r,\varphi, z) =
	\begin{pmatrix}
		r \cos(\varphi) \\
		r \sin(\varphi) \\
		z
	\end{pmatrix}
\]
Da die $r$-Komponente in Zylinderkoordinaten den Abstand von der $z$-Achse
beschreibt folgt aus dem Satz des Pytaghoras:
\[
	r = \sqrt{x^2 + y^2}
\]
Dann lässt sich für $\varphi$ und $z$ bestimmen:
\[
	\begin{pmatrix} x \\ y \\ z \end{pmatrix}
	=
	\begin{pmatrix} r\cos(\varphi) \\ r\sin(\varphi) \\ z \end{pmatrix}
	=
	\begin{pmatrix} 
		\sqrt{x^2 + y^2} \cos(\varphi) \\ 
		\sqrt{x^2 + y^2} \sin(\varphi) \\ 
		z 
	\end{pmatrix}
	\Rightarrow
	\varphi 
	= \arccos\left(\frac{x}{\sqrt{x^2+y^2}}\right)
	=\arcsin\left(\frac{y}{\sqrt{x^2+y^2}}\right)
\]
Die zwei Ausdrücke für $\varphi$ bedeuten das natürlich 
gleiche; $x$, $y$, und
$\sqrt{x^2 + y^2}$ sind Anka-, Gegegenkathede und Hypothenuse und der 
Quotient von diesen sind ja gerade sin und cos. \\
\hspace{0.2cm}
\\
Durch scharfes hingucken erkennt man für $z$:
\[
	z = z
\]
Damit lautet die Umkehrabbildung, bei der mindestens $x$ oder $y$ ungleich
0 sein müssen:
\[
	\vec \phi^{-1}(x,y,z)
	=
	\begin{pmatrix}
		r \\ \varphi \\ z
	\end{pmatrix}
	=
	\begin{pmatrix}
		\sqrt{x^2 + y^2} \\
		\arcsin\left(\frac{y}{\sqrt{x^2+y^2}}\right) \\
		z
	\end{pmatrix}
\]


\end{document}
