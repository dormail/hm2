\documentclass[10pt]{article}
\usepackage[utf8]{inputenc}
\usepackage[margin=2.5cm]{geometry}

% packages die oefters gebraucht werden
\usepackage{amsmath, amssymb}
\usepackage{graphicx}
\usepackage{fancyhdr}
\usepackage{aligned-overset}
\usepackage{float}
\usepackage{wrapfig}

\fancyhf{}
% vspaces in den headern fuer Distanzen notwendig
% linke Seite: Namen der Abgabegruppe
\lhead{\textbf{Matthias Maile}\vspace{0.5cm}}
% rechte Seite: Modul, Gruppe, Semester
\rhead{\textbf{Höhere Mathematik II\\Sommersemester 2020}\vspace{0.5cm}}
% Center: nr. des blattes
\chead{\vspace{1.5cm}\huge{\textbf{7. Übungsblatt}}}
% benoetigt damit der eigentliche Text nicht in der Überschrift steckt
\setlength{\headheight}{3cm} 

% custom commands
\newcommand{\arsinh}{\text{arsinh}}

% for newpage to reset headheight
\newcommand{\secondpage}{
	\newpage 
	\setlength{\headheight}{0cm}
}

\begin{document}
\thispagestyle{fancy}
\section*{Aufgabe 25}
Der Abstand zum Punkt $\vec p$ ist gegeben durch
\[ d(x,y,z) = \sqrt{x^2 + y^2 + (z-3)^2} \]
Da dieser minimiert bzw. maximiert werden soll, kann, aufgrund der Monotonie der Wurzelfunktion, zur Berechnung
von Extrema auch die Wurzel weggelassen werden:
\[ d^\prime (x,y,z) = x^2 + y^2 + (z-3)^2 \]
Die Nebenbedingung ist dabei, dass der Punkt auf dem Ellipsoid liegen soll:
\[ E(x,y,z) = x^2 + 2y^2 + 4z^2 - 8 = 0 \]
Dann folgt aus dem notwendigen Kriterium ein Extremum bei $\vec a$, sofern
\[ \nabla d^\prime(\vec a) - \lambda \nabla E(\vec a) = \vec 0, \ E(\vec a) = 0 \quad 
\text{und} \quad \nabla E (\vec a) \neq \vec 0 \]
erfüllt sind.
Die rechte Gleichung liefert uns als einzigen Ausnahmepunkt den Ursprung, welcher nicht auf der Ellipse liegt
(und damit nicht weiter beachtet werden muss):
\[ \nabla E (\vec a) = (2x, \ 4y, \ 8z) \overset!= (0,0,0) \Rightarrow \nabla E(\vec 0) = \vec 0 \]
Damit lässt sich das Gleichungssystem aufstellen:
\begin{align*}
	\nabla d^\prime(\vec a) - \lambda \nabla E(\vec a) = \vec 0
	\Rightarrow \hspace{2cm}
	2x - \lambda 2x &= 0 \\
	2y - \lambda 4y &= 0 \\
	2x - 6 - \lambda 8z &= 0 \\
	x^2 + 2y^2 + 4z^2 - 8 &= 0
\end{align*}
Die ersten 3 Zeilen liefern 3 Kandidaten für $\lambda$:
\[
	\lambda_1 = 1 \quad
	\lambda_2 = \frac12 \quad
	\lambda_3 = \frac{-3}{4z} + \frac14
\]
Damit lassen sich 3 Fälle aufstellen:\\
1. Fall: $\lambda_3 = \lambda_1, \ y = 0, \ x$ variabel
\begin{align*}
	\lambda_1 = 1 \overset{!}&{=} \frac{-3}{4z} + \frac14 \\
	\Rightarrow z &= -1
\end{align*}
2. Fall: $\lambda_3 = \lambda_2 \ x = 0, \ y$ variabel
\begin{align*}
	\lambda_2 = \frac12 \overset{!}&{=} \frac{-3}{4z} + \frac14 \\
	\Rightarrow z &= -3
\end{align*}
Da dies gegen unsere Randbedingung verstößt (liegt nicht auf dem Parabolloid), ist dies keine Lösung. \\
3. Fall: $x = y = 0, \ z$ variabel
\begin{align*}
	E(0,0,z) = 4z^2 - 8 &= 0 \\
	\Leftrightarrow 
	z^2 &= 2 \quad \Rightarrow z = \pm \sqrt 2
\end{align*}

\secondpage

Für den 1. Fall lässt sich $x$ bestimmen:
\[ E(x,0,-1) = x^2 + 4 - 8 = 0 \Rightarrow x = \pm 2 \]

Damit existieren Mehrere mögliche Extrema:
\[
	\vec a_{1/2} = \begin{pmatrix} \pm 2 \\ 0 \\ -1 \end{pmatrix} \quad
	\vec a_{3/4} = \begin{pmatrix} 0 \\ 0 \\ \pm \sqrt 2 \end{pmatrix}
\]
Setzen man diese in unsere Distanzformel $d(x,y,z)$ ein, erhalten wir die Werte:
\begin{align*}
	d_1 &= d(\vec a_1) = \sqrt{20} \\
	d_2 &= d(\vec a_2) = \sqrt{20} \\
	d_3 &= d(\vec a_3) \approx 1.59 \\
	d_4 &= d(\vec a_4) = \sqrt{8}
\end{align*}
Somit liegt die Minimale Distanz $d_\text{min} \approx 1.59$ am 
Punkt $(0,0,\sqrt 2)$, und die maximale Distanz $d_\text{max} = \sqrt{20}$ an den
Punken $(\pm2, 0, -1)$ vor.


\section*{Aufgabe 26}
a) Damit das Integral berechnet werden kann, teile ich es in einen Bereich A $0 \leq x < 1$ und einen Zweiten
B $1 \leq x \leq 2$ auf.
\begin{align*}
	\int_D \frac{y^2}{x^2} d(x,y) 
	% aufteilen der flaeche
	&= \int_A \frac{y^2}{x^2} d(x,y) +\int_B \frac{y^2}{x^2} d(x,y) \\
	% integrationsgrenzen
	&= \int_0^1 \int_0^x \frac{y^2}{x^2} \ dy \ dx
	+ \int_1^2 \int_0^{\frac1x} \frac{y^2}{x^2} \ dy \ dx \\
	% nac y integrieren
	&= \int_0^1 \frac1{x^2} \left[ \frac{y^3}{3} \right]_0^x \ dx
	+ \int_1^2 \frac1{x^2} 
	\left[ \frac{y^3}{3} \right]_0^{\frac1x} \ dx \\
	% y einsetzen
	&= \int_0^1 \frac1{x^2}\frac{x^3}{3} \ dx
	+ \int_1^2 \frac1{x^2} 
	\frac{1}{3x^3} \ dx \\
	% vereinfachen
	&= \int_0^1 \frac x3 \ dx
	+ \int_1^2 \frac{1}{3x^5} \ dx \\
	% nach x integrieren
	&= \left[ \frac{x^2}{6} \right]_0^1
	+ \frac13 \left[ \frac{-1}{4x^4} \right]_1^2 \\
	% einsetzen
	&= \frac16 + \frac 1{12} - \frac{1}{12} \frac1{16} \\
	% ausrechen
	&= \frac{47}{16} \frac{1}{12}
\end{align*}

b) Nach dem Satz von Fubini dürfen die Integrale vertauscht werden.
\begin{align*}
	\int_G (x+y+z)d(x,y,z)
	&= \int_0^{xy} \int_x^1 \int_0^1 (x+y+z) \ dx dy dz  \\
	% integrale vertauschen
	&= \int_0^1 \int_x^1  \int_0^{xy} (x+y+z) \ dz dy dx  \\
	% nach z integrieren
	\text{(HS)} \Rightarrow
	&= \int_0^1 \int_x^1 \left[zx + zy + \frac12 z^2 \right]_0^{xy} dy dx  \\
	% werte fuer z einsetzen
	&= \int_0^1 \int_x^1 \left[x^2y + xy^2 + \frac12 x^2y^2 \right] dy dx  \\
	% nach y integrieren
	\text{(HS)} \Rightarrow
	&= \int_0^1 \left[\frac12 x^2y^2 + \frac13 xy^3 + \frac16 x^2y^3 \right]_x^1  dx  \\
	% werte fuer y einsetzen
	&= \int_0^1 \left[\frac12 x^2 + \frac13 x + \frac16 x^2 
	- \frac12 x^4 + \frac13 x^4 + \frac16 x^5 \right]  dx  \\
	% vereinfachen
	&= \int_0^1 \left[\frac16 x^5 - \frac16 x^4   + \frac23 x^2 + \frac13 x \right]  dx  \\
	% nach x integrieren
	\text{(HS)} \Rightarrow
	&= \left[\frac1{36} x^6 - \frac1{30} x^5   + \frac2{9} x^3 + \frac16 x^2 \right]_0^1 \\
	% einsetzen
	&= \frac1{36} - \frac1{30} + \frac2{9} + \frac16 \\
	% vereinfachen
	&= \frac9{36} + \frac4{30} \\
	&= \frac14 + \frac2{15}
\end{align*}

\end{document}
